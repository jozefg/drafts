\documentclass{amsart}
\title{A Weird Result of Church's Thesis}
\author{Danny Gratzer}
\newcommand{\uni}{\mathcal{U}}
\newcommand{\nat}{\mathbb{N}}
\begin{document}
\maketitle

Church's law is a widely accepted principle in computer science. It
roughly states that any ``real'' computable function may be realized
by a program in the lambda calculus. Church's law is not on equal
footing to any theorem, in order to do this we'd have to select some
particular formal model of computation and then we have the same
problem: what if the model we chose is excluding some reasonable
computable function.

In this note I'd like to highlight a surprising realization I had
about Church's law, if we internalize it into a system then that
system may define some frightening observations that seem to fly in
the face of the standard model that we're trying to construct. To pick
a specific system, let's work with a standard computational type
theory with the type formers you'd expect. Church's law may then be
stated that as follows

\[
  \mathtt{church} :
  \exists f : \nat \to (\nat \to \nat).
  \ \Pi g : \nat \to \nat.
  \ \exists n : \nat.
  \ f\ n =_{\nat \to \nat} g
\]

This states that there's a computable, surjective function from $\nat$
to functions on the naturals. Intuitively, we connect this to Church's
law by viewing the $\nat$ that $f$ takes as a G\"odel numbered
lambda term.

\end{document}
