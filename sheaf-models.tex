\documentclass[12pt]{amsart}
\usepackage{amsmath, amsthm}
\usepackage{times, fullpage, microtype}
\usepackage{minted}
\usepackage{mathpartir}

% Sets
\newcommand{\cantor}{\mathbb{C}} % Is there a standard symbol for the Cantor space?

% PL notation
\newcommand{\lam}[3]{\lambda #1 : #2.\, #3}
\newcommand{\ap}[2]{#1\ #2}
\newcommand{\prl}[1]{\mathtt{fst}(#1)}
\newcommand{\prr}[1]{\mathtt{snd}(#1)}
\newcommand{\natrec}[3]{\mathtt{rec}(#1;\ #2;\ #3)}
\newcommand{\nat}{\mathtt{nat}}
\renewcommand{\succ}[1]{\mathtt{succ}(#1)}

% amsthm config
\newtheorem{thm}{Theorem}[section]
\newtheorem{cor}[thm]{Corollary}
\newtheorem{lem}[thm]{Lemma}

% Preamble
\title{On the Fan Theorem}
\author{Daniel Gratzer}
\date{\today}
\begin{document}
\maketitle

\section{Introduction}

In this post I'd like to talk a little about the fan theorem and type
theory. Now before this scares the casual reader off, let me say that
I'm only going to be working with a simple total functional language
and if you ignore the math you end up with a nice justification for
the slogan ``computation is continuous''.

Now the origin of the fan theorem traces back to Browerian
mathematics. Specifically, far before things like the propositions as
types correspondence was being contemplated, there was a (small) group
of mathematicians working on studying what math was possible
intuitionistically. It's unfortunately quite hard to find lucid
explanations of that sort of work\footnote{I recommend Elements of
  Intuitionism or Constructivism in Mathematics} but one of the main
results of this study was a large mathematical theory surrounding
sequences of numbers or binary digits. Most of this is too far afield
of what I want to talk about to delve into but one important theorem
dealt with what are called \emph{spreads} on binary sequences. These
are really just collections of prefixes to infinite binary sequences,
we can think of them accordingly as predicates on lists of bits. The
fan theorem concerns such predicates that are said to \emph{bar} the
space of binary sequences. All this means is that for any infinite
binary sequence $\alpha$, there is an $n$ so that $P(\alpha[0:n])$ is
true. In words, every infinite sequence contains a prefix that $P$
contains. We are now in a position to state the fan theorem.
\begin{thm}[Fan Theorem]\label{thm:intro:fan1}
  If a given predicate $P$ on binary sequences is \emph{decidable} and
  \emph{bars} the total space of binary sequences, then there is a
  number $M$ so that for any infinite sequence $\alpha$, there is an
  $n < M$ so that $P(\alpha[0:n])$ holds.
\end{thm}
This is a rather difficult to parse claim so let's take a second to
understand what it's getting at. The gist of it is that a given
predicate which acts as a bar and is decidable can be subjected to a
far stronger statement, not only must it be that all infinite
sequences have a prefix in $P$, but these prefixes can all be chosen
within a suitable length.

Let's take a moment to step back and see what this entails
mathematically. A spread, being a collection of finite sequences, is
in fact understandable as a collection of \emph{open sets} for the
Cantor space. One presentation of the Cantor sequence involves a
particular subset of the unit interval. A more intuitive presentation
is that the Cantor space is compromised of infinite binary sequences
where basic open sets are particular, finite, prefixes. This means
that a bar is precisely an open cover in topological terms. Phrased in
this way, we could have equivalently said
\begin{thm}[Fan Theorem]\label{thm:intro:fan2}
  Every basic open cover the Cantor space has a subcover where each open set
  is a prefix of length less than $M$
\end{thm}
\begin{cor}
  Every basic open cover of the Cantor space has a finite subcover.
\end{cor}
\begin{proof}
  Supposing we have an open cover, we can apply
  theorem~\ref{thm:intro:fan2} to conclude that there is a subcover
  where each open set is a prefix of length less than $M$ for some
  $M$. However, there are only\footnote{In mathematics, exponential
    time is just fine after all.} $2^M$ such open sets so it must be
  that the open cover is finite.
\end{proof}
\begin{cor}
  Every open cover of the Cantor space has a finite subcover
\end{cor}
\begin{proof}
  We recall that a full open cover is one constructed of from
  arbitrary unions of prefixes. Consider the set of all the prefixes
  in all such unions. This is clearly a cover if the original set
  was. As a basic open cover, we can find a finite subcover for it by
  the previous corollary. Now consider the finite subset of the
  original cover which contains the unions of prefixes containing the
  prefixes in the finite subcover just constructed. This is clearly
  finite and as it is strictly larger than a cover, it too must be
  a cover. Thus we have our desired finite subcover.
\end{proof}

This means that topologically, the Fan theorem is the more familiar
statement that the Cantor space is compact. Can we derive something
else useful from this? Well one thing we notice is that if we have a
function $\cantor \to \{0, 1\}$ (Here we have the discrete topological
space $\{0, 1\}$) we know that $f^{-1}(\{0, 1\}) = \cantor$. However,
we can decompose this to say that given $U_0 = f^{-1}(\{0\})$ and
$U_1 = f^{-1}(\{1\})$ then $U_0 \cup U_1 = \cantor$. Now we have an
open cover because we know that $f^{-1}(\{i\})$ is open by simple
definition of continuity. Let's further decompose $U_i$ as the union
of basic open sets, $U_0 = \bigcup_j B_j$ and $U_1 = \bigcup_j P_j$.
Now the existence of a finite subcover means that there is a finite
collection $B_{j_1}, B_{j_2}, ..., P_{j_n}, B_{j_{n + 1}}, ...$ so
that these basic open sets cover the entirety of $\cantor$. This can
be stated in two different ways,

We can phrase this in a way that might be more familiar to
programmers. Doing so requires the observation that sequences of
binary numbers might as well be functions from $\mathbb{N}$ to
$\{0, 1\}$. In more PLish terms, we can encode them as
{\tt nat -> bool}. Now does a spread become? Well since it's merely a
predicate on binary sequences we can encode it simply as a function
{\tt (nat -> bool) -> nat}.


Before we can get to technical questions like exhibiting a concrete
model, we need to fix our language and the rules for it. I don't want
to work in full dependent type theory because it's a bit too
complicated to work with. Instead I shall work with the more modest
system called G\"odel's System T. This language is nice and
straightforward, it includes functions, natural numbers, pairs and that's
it. We can write down the rules for the language quite quickly
\begin{align*}
  \ap{(\lam{x}{\tau}{e})}{v} \mapsto& [v/x]e\\
  \prl{(v_1, v_2)} \mapsto& v_1\\
  \prr{(v_1, v_2)} \mapsto& v_2\\
  \natrec{0}{e_1}{x.e_2} \mapsto& e_1\\
  \natrec{\succ{v}}{e_1}{x.e_2} \mapsto& [\natrec{v}{e_1}{x.e_2}/x]e_2\\
\end{align*}
along with the expected compatibility rules. This can be summarized
for people familiar with functional programming by just saying that we
have tuples, along with {\tt fst} and {\tt snd} and a fold-like
operation on natural numbers. Natural numbers are represented by
saying that we have $\succ{-}$ and $0$, $\succ{n}$ should be read as saying
``one more than the number {\tt n}'' and 0 is zero. If we were to
define {\tt rec} in Haskell, it would look something like
\begin{minted}{haskell}
  foldnat :: Nat -> a -> (a -> a) -> a
  foldnat 0 z s = z
  foldnat n z s = s (foldnat (n - 1) z s)
\end{minted}
If you're curious, this is exactly what {\tt foldr} looks like if we
restrict ourselves to {\tt [()]} which is exactly the terrible
representation of natural numbers we've picked for our language!

We'd now like to state Markov's principle in a way that's compatible
with System T since System T entirely lacks the dependent types we
used to write it down before. What we shall instead say is that
Markov's principle can be used to decide a much more restricted
property, the question of whether or not a given function
$\nat \to \nat$ is ever $0$. That is, we want to show that we cannot
code up the following function
\section{An Intuition for Independence}
\section{Sketching Sheaves}
\section{Independence Result}
\section{Conclusions}

\end{document}
